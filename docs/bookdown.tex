\PassOptionsToPackage{unicode=true}{hyperref} % options for packages loaded elsewhere
\PassOptionsToPackage{hyphens}{url}
%
\documentclass[12pt,]{report}
\usepackage{lmodern}
\usepackage{amssymb,amsmath}
\usepackage{ifxetex,ifluatex}
\usepackage{fixltx2e} % provides \textsubscript
\ifnum 0\ifxetex 1\fi\ifluatex 1\fi=0 % if pdftex
  \usepackage[T1]{fontenc}
  \usepackage[utf8]{inputenc}
  \usepackage{textcomp} % provides euro and other symbols
\else % if luatex or xelatex
  \usepackage{unicode-math}
  \defaultfontfeatures{Ligatures=TeX,Scale=MatchLowercase}
\fi
% use upquote if available, for straight quotes in verbatim environments
\IfFileExists{upquote.sty}{\usepackage{upquote}}{}
% use microtype if available
\IfFileExists{microtype.sty}{%
\usepackage[]{microtype}
\UseMicrotypeSet[protrusion]{basicmath} % disable protrusion for tt fonts
}{}
\IfFileExists{parskip.sty}{%
\usepackage{parskip}
}{% else
\setlength{\parindent}{0pt}
\setlength{\parskip}{6pt plus 2pt minus 1pt}
}
\usepackage{hyperref}
\hypersetup{
            pdftitle={Relaciones topológicas entre los conjuntos Multibrots y de Julia generados por el polinomio \{z\^{}p+c\}},
            pdfauthor={MALLQUI BAÑOS Ricardo Michel},
            pdfborder={0 0 0},
            breaklinks=true}
\urlstyle{same}  % don't use monospace font for urls
\usepackage{longtable,booktabs}
% Fix footnotes in tables (requires footnote package)
\IfFileExists{footnote.sty}{\usepackage{footnote}\makesavenoteenv{longtable}}{}
\usepackage{graphicx,grffile}
\makeatletter
\def\maxwidth{\ifdim\Gin@nat@width>\linewidth\linewidth\else\Gin@nat@width\fi}
\def\maxheight{\ifdim\Gin@nat@height>\textheight\textheight\else\Gin@nat@height\fi}
\makeatother
% Scale images if necessary, so that they will not overflow the page
% margins by default, and it is still possible to overwrite the defaults
% using explicit options in \includegraphics[width, height, ...]{}
\setkeys{Gin}{width=\maxwidth,height=\maxheight,keepaspectratio}
\setlength{\emergencystretch}{3em}  % prevent overfull lines
\providecommand{\tightlist}{%
  \setlength{\itemsep}{0pt}\setlength{\parskip}{0pt}}
\setcounter{secnumdepth}{5}

% set default figure placement to htbp
\makeatletter
\def\fps@figure{htbp}
\makeatother

\usepackage{etoolbox}
\makeatletter
\providecommand{\subtitle}[1]{% add subtitle to \maketitle
  \apptocmd{\@title}{\par {\large #1 \par}}{}{}
}
\makeatother
\usepackage[spanish,es-noquoting]{babel}
\usepackage[utf8]{inputenc}
\usepackage{amsfonts}
\usepackage[T1]{fontenc}
\usepackage{booktabs,longtable}
\usepackage[referable]{threeparttablex}
\usepackage{dsfont}
\usepackage[a4paper]{geometry}
\geometry{verbose,tmargin=2.5cm,bmargin=2.5cm,lmargin=3cm,rmargin=2.5cm}
\usepackage{booktabs}
\renewcommand{\rmdefault}{ptm}
%\usepackage[lite,subscriptcorrection,nofontinfo,zswash]{mtpro2}
\usepackage{textcase}
\usepackage{makecell}
\usepackage{lscape}
\usepackage{multirow}
\usepackage{tabto}
\usepackage{bigstrut}
\usepackage{titlesec}
\DeclareFontFamily{U}{mt2ms}{\skewchar\font42}%
\DeclareFontShape{U}{mt2ms}{m}{n}{<-7>mt2mcf<7-9>mt2mcs<9->mt2mct}{}%
\DeclareFontShape{U}{mt2ms}{m}{it}{<-7>mt2msf<7-9>mt2mss<9->mt2mst}{}%
\DeclareFontShape{U}{mt2ms}{b}{it}{<-7>mt2bmsf<7-9>mt2bmss<9->mt2bmst}{}%


\titlespacing*{\chapter}
{0pt} {5em plus .01em minus .01em} {0em plus .01em minus .01em}
\titlespacing*{\section}
{0pt} {0em plus .01em minus .01em} {0em plus .01em minus .01em}
\titlespacing*{\subsection}
{0pt} {0em plus .01em minus .01em} {0em plus .01em minus .01em}
\titlespacing*{\subsubsection}
{0pt} {0em plus .01em minus .01em} {0em plus .01em minus .01em}
\titlespacing*{\paragraph} {0pt}{1.25ex plus 1ex minus .2ex}{2em}
\titlespacing*{\subparagraph} {\parindent}{3.25ex plus 1ex minus .2ex}{1em}
\titleformat*{\section}{\normalsize\bfseries}
\titleformat*{\subsection}{\normalsize\bfseries}
\setlength{\parskip}{0.7em plus .01em minus .01em}

%%%%%%%%%%%%%%%%%%%%%%%%%%
\usepackage{xpatch}%space equation
\xapptocmd\normalsize{%
\abovedisplayskip=0.5em plus 0.01em minus 0.01em
\abovedisplayshortskip=0.5em plus 0.01em minus 0.01em
\belowdisplayskip=0.5em plus 0.01em minus 0.01em
\belowdisplayshortskip=0.5em plus 0.01em minus 0.01em
}{}{}
%%%%%%%%%%%%%%%%%%%%%


\renewcommand{\thechapter}{\Roman{chapter}}

\titleformat{\chapter}[display]
{\normalfont\normalsize\bfseries\centering}{CAPÍTULO \thechapter}{0.0em}{\normalsize\bfseries}
%\renewcommand{\thechapter}{\Roman{chapter}}
\renewcommand{\thesection}{\arabic{chapter}.\arabic{section}}
\renewcommand{\thesubsection}{\arabic{chapter}.\arabic{section}.\arabic{subsection}}
\renewcommand{\thesubsubsection}{\arabic{chapter}.\arabic{section}.\arabic{subsection}.\arabic{subsubsection}}
% https://github.com/rstudio/rmarkdown/issues/337
\let\rmarkdownfootnote\footnote%
\def\footnote{\protect\rmarkdownfootnote}

% https://github.com/rstudio/rmarkdown/pull/252
\usepackage{titling}
\setlength{\droptitle}{-2em}

\pretitle{\vspace{\droptitle}\centering\huge}
\posttitle{\par}

\preauthor{\centering\large\emph}
\postauthor{\par}

\predate{\centering\large\emph}
\postdate{\par}
\usepackage[]{natbib}
\bibliographystyle{apalike}

\title{Relaciones topológicas entre los conjuntos Multibrots y de Julia generados por el polinomio \({z^p+c}\)}
\author{MALLQUI BAÑOS Ricardo Michel}
\date{2020-02-15}

\begin{document}
\maketitle

{
\setcounter{tocdepth}{1}
\tableofcontents
}
\newcommand{\N}{\mathbb{N}}
\newcommand{\R}{\mathbb{R}}
\newcommand{\CC}{\mathbb{C}}
\newcommand{\I}{\mathbb{I}}
\newcommand{\f}{\mathbb{f}}
\newcommand{\X}{\mathbb{X}}
\newcommand{\D}{\mathbb{D}}
\newcommand{\Z}{\mathbb{Z}}
\newcommand{\Q}{\mathbb{Q}}

\newcommand{\norm}[1]{\left\Vert#1\right\Vert}
\newcommand{\abs}[1]{\left\vert#1\right\vert}
\newcommand{\set}[1]{\left\{#1\right\}}
\newcommand{\seq}[1]{\left<#1\right>}
\newcommand{\co}[1]{\left[#1\right]}
\newcommand{\cc}[1]{\left(#1\right)}

\newcommand{\J}{\mathcal{J}}
\newcommand{\K}{\mathcal{K}}
\newcommand{\M}{\mathcal{M}}
\newcommand{\F}{\mathcal{F}}

\hypertarget{we}{%
\chapter{Preliminares}\label{we}}

\hypertarget{topologuxeda-de-los-nuxfameros-complejos}{%
\section{Topología de los números complejos}\label{topologuxeda-de-los-nuxfameros-complejos}}

\hypertarget{espacio-muxe9trico}{%
\subsection{Espacio métrico}\label{espacio-muxe9trico}}

\citep{Barnsley} \((X,d)\), espacio métrico

\begin{itemize}
\tightlist
\item
  \(d(a,b)\geq0\).
\item
  \(d(a,b)=0\Longleftrightarrow a=b\).
\item
  \(d(a,b)=d(b,a)\) (axioma de simetría).
\item
  \(d(a,b)\leq d(a,c)+d(c,b)\) (desigualdad triangular)
\end{itemize}

Para todo \(a,b\)
y \(c\) en \(X\).

\citep{articlechurchil} se tiene que \(\mathbb{C}\)
\[d(z,w)=\sqrt{\left(x_{1}-x_{2}\right)^{2}+\left(y_{1}-y_{2}\right)^{2}},\]
\protect\hyperlink{conjunto}{}

\hypertarget{conjunto}{%
\subsection{Conjuntos compactos}\label{conjunto}}

Sea \(A\) subconjunto de un espacio métrico \(X\), decimos que \(A\)
es compacto si para cada cubierta abierta \(\mathcal{C}\) de \(A\) existe
una subcubierta finita \(\mathcal{C}'.\)

\hypertarget{conjuntos-conexos}{%
\subsection{Conjuntos conexos}\label{conjuntos-conexos}}

Decimos que el espacio \(X\) es conexo
si, para todos los conjuntos abiertos \(U\) y \(V\) en \(X\), no
vacíos, de modo que \(X\subset U\cup V\), se tiene \(U\cap V\neq\emptyset\).
En otras palabras, \(X\) no puede ser cubierto por dos conjuntos abiertos
disjuntos.

\begin{enumerate}
\def\labelenumi{\arabic{enumi}.}
\item
  Conexidad por trayectorias
\item
  Conjuntos convexos
\item
  Componentes conexas
\end{enumerate}

\hypertarget{sucesiones}{%
\subsection{Sucesiones}\label{sucesiones}}

Una sucesión en un conjunto \(X\) es una función de los naturales en
\(X\). Designamos por \(x_{n}\) al \(n\) -- ésimo término de la sucesión
es decir, a la imagen de \(n\) a través de la función \(\left\{x_{n}\right\}_{n\in\mathbb{N}}\).

Decimos que una sucesión \(\left\{x_{n}\right\}\) en un espacio métrico \(({X,d})\)
es convergente en \(X\) si existe un punto \(p\in X\) con la propiedad
de que \(\forall\epsilon\geq0,\) \(\exists N\in\mathbb{N}:\forall n\geq N,\)
\(d(p,x_{n})<\epsilon\).

\hypertarget{el-plano-extendido-complejo}{%
\subsection{El plano extendido complejo}\label{el-plano-extendido-complejo}}

En esta sección se recopilará información acerca de la topología del
plano complejo asociado al número \(\infty\) siendo esto posible con
la ayuda de la esfera de Riemann. De acuerdo a \citep{rudin} el conjunto
\(\mathbb{C}_{\infty}=\mathbb{C}\cup\left\{\infty\right\}\) se llama \emph{plano complejo
ampliado}, (\(\infty\notin\mathbb{C}\)). En \(\mathbb{C}\) adoptamos la aritmética
usual con \(\infty\). \(\mathbb{C}_{\infty}\) se le dota de una topología para
la cual es un espacio compacto, es decir los entornos de los puntos
de \(a\in\mathbb{C}\) son por definición los entornos en \(\mathbb{C}_{\infty}\) y
para \(a=\infty\) una base de entornos en \(\mathbb{C}_{\infty}\) viene dada
por \(V_{r}=\left\{z\in\mathbb{C};\left\vert z\right\vert>r\right\}\cup\left\{\infty\right\},\) tomando todos
los \(r>0\). \(\mathbb{C}\) se puede identificar con
\[S=\left\{(x,y,z)\in {\mathbb{R}^{3}}; x^{2}+y^{2}+z^{2}=1\right\}.\]

\hypertarget{topologuxeda-de-hausdorff}{%
\subsection{Topología de Hausdorff}\label{topologuxeda-de-hausdorff}}

Sea \((X,d)\) un espacio métrico completo. Entonces \(\mathfrak{H}(X)\)
denota el espacio cuyos puntos son los subconjuntos compactos no vacíos
de \(X.\)
Sea \(x\in X\), \(A\) y \(B\in\mathfrak{H}(X)\).
\[
d(x,B)=\min\left\{d(x,y):y\in B\right\}.
\]
\[
d(A,B)=\max\left\{d(x,B):x\in A\right\}.
\]
\[
h(A,B)=d(A,B)\vee d(B,A).
\]

\begin{figure}

{\centering \includegraphics[width=0.5\linewidth]{ww} 

}

\caption{Distancia de Housdorff}\label{fig:ww}
\end{figure}

\protect\hyperlink{compleja}{}

\hypertarget{compleja}{%
\section{Trasformaciones de variable compleja}\label{compleja}}

\hypertarget{transformaciuxf3n-de-muxf6bius}{%
\subsection{Transformación de Möbius}\label{transformaciuxf3n-de-muxf6bius}}

las transformaciones de Möbius \(T:\mathbb{C}_{\infty}\rightarrow\mathbb{C}_{\infty}\)
no constantes, son definidas mediante funciones racionales de la forma
\[
T(z)=\frac{az+b}{cz+d},
\]
donde \(a,b,c,d\in\mathbb{C}_{\infty},\) \(ad-bc\neq0\) donde se utilizan
las convenciones funcionales habituales \(T(\infty)=a/c\) además \(T(-d/c)=\infty\)
si \(c\neq0\) \(T(\infty)=\infty\) si \(c=0.\) Para cada \(w\in\mathbb{C}_{\infty}\)
la ecuación \(T(z)=w\) tiene una única solución
\[
z=T^{-1}(w)=\frac{dw-b}{-cw+a}
\]
luego \(T^{-1}\) sigue siendo una transformación del mismo tipo.
\protect\hyperlink{holomorfa}{}

\hypertarget{holomorfa}{%
\subsection{Transformaciones holomorfas y analíticas}\label{holomorfa}}

Dado un dominio \(\Omega\) y \(f:\Omega\rightarrow\mathbb{C}\)
una función, decimos que \(f\) es holomorfa en \(z_{0}\in\Omega\) si
el cociente
\[
\frac{f(z_{0}+h)-f(z_{0})}{h}
\]
converge a un único límite cuando \(h\rightarrow0\) (donde \(h\in\mathbb{C},h\neq0\)
y \(z+h\in\Omega\)). Si el limite existe, entonces decimos que la derivada
(compleja) de \(f\) en \(z_{0}\) esta dada por
\[
f'(z_{0})=\lim_{h\rightarrow\infty}\frac{f(z_{0}+h)-f(z_{0})}{h}.
\]
\protect\hyperlink{conforme}{}

\hypertarget{conforme}{%
\subsection{Transformaciones conformes}\label{conforme}}

Una transformación conforme es una función \(f:A\rightarrow\mathbb{C},\)
diferenciable en \({z_{0}\in A\subset\mathbb{C}},\) que preserva el
ángulo que dos curvas \(\alpha\colon[a,b]\rightarrow A\) y \(\beta\colon[a,b]\rightarrow A,\)
diferenciables en \({\alpha^{-1}\,(z_{0})}\) y \({\beta^{-1}\,(z_{0})},\)
respectivamente, forman entre sí en \(z_{0}\). Es decir \(f\) es conforme
en \(z_{0}\) cuando se verifica
\[
\arg\left[\frac{\left(f\circ\alpha\right)'(z_{0})}{\left(f\circ\beta\right)'(z_{0})}\right]=\arg\left(\frac{\alpha'(z_{0})}{\beta'(z_{0})}\right),
\]
siempre y cuando \(\alpha'(z_{0})\) y \({\beta'(z_{0})\,}\) sean vectores
tangentes no nulos. \protect\hyperlink{conjugac}{}

\hypertarget{conjugac}{%
\subsection{Conjugación analítica}\label{conjugac}}

Sean \(f\) y \(g\) transformaciones complejas, de acuerdo a \citep{Carleson}:
Decimos que una función \(f\) : \(U\rightarrow U\) es conformalmente
conjugado a una función \(g:V\rightarrow V\) si existe una transformación
conforme \(\varphi:U\rightarrow V\) tal que \(g=\varphi\circ f\circ\varphi^{-1}\),
es decir, tal que
\[
\varphi(f(z))=g(\varphi(z))
\]
Las transformaciones \(f\) y \(g\) se pueden considerar como las mismas
transformaciones visto en diferentes sistemas de coordenadas \protect\hyperlink{ww}{C2}

\hypertarget{ww}{%
\chapter{Dinamica discreta compleja}\label{ww}}

\hypertarget{iteraciuxf3n-de-funciones-complejas}{%
\section{Iteración de funciones complejas}\label{iteraciuxf3n-de-funciones-complejas}}

Se define
como la \textbf{composición reiterada} de \(f\) consigo misma \(k\) veces y la
denotamos por \(f^{k}\) donde
\(f^{k+1}=f\circ f^{k}.\)
Ademas \(f^{1}=f\) y \(f^{0}=id.\) La órbita de \(x\) bajo \(f\) como la \textbf{sucesión}
\(x,f(x),f^{2}(x),f^{3}(x),\ldots,f^{k}(x),\ldots\)

Dado un punto periódico \(z_{0}\) con periodo \(k,\) se define el número
\textbf{\(\lambda\)}, llamado el \textbf{valor propio} o \textbf{el multiplicador}
\[
\lambda=\begin{cases}
\left(f^{k}\right)'(z_{0})=\prod_{i=0}^{k-1}f'\left(z_{i}\right) & \mbox{si}\ z_{0}\neq\infty\\
\frac{1}{\left(f^{k}\right)'(z_{0})} & \mbox{si}\ z_{0}=\infty.
\end{cases}
\]

\(z_{0}\) es un \textbf{punto fijo} de \(f\) si verifica la ecuación
\textbf{\(f(z_{0})=z_{0}.\)}
Un punto fijo \(z\) de \(f\), es periódica, de periodo 1.

\begin{itemize}
\item
  Puntos eventualemente periódicos
\item
  Puntos eventualmente periódico-repelentes
\item
  Cuencas superactractoras (inmediata)
  \protect\hyperlink{fatou}{}
\end{itemize}

\hypertarget{fatou}{%
\section{\texorpdfstring{Conjuntos de Fatou y Julia \(\mathcal{K}\), \(\mathcal{J}\) y \(\mathcal{F}\)}{Conjuntos de Fatou y Julia \textbackslash{}mathcal\{K\}, \textbackslash{}mathcal\{J\} y \textbackslash{}mathcal\{F\}}}\label{fatou}}

El \textbf{conjunto lleno de Julia} asociado a
una función \(f\): \(\mathcal{k}=\mathcal{K}(f)\) esta formado por \(z\in\mathbb{C}_\infty\), donde la secuencia definida \(z_{0}=z,z_{n+1}=f\left(z_{n}\right)\),
(órbita positiva) de \(z,\) \textbf{no converge a infinito}; y el \textbf{conjunto de
Julia} es la frontera del \textbf{conjunto lleno de Julia} es decir \(\mathcal{J}=\partial\mathcal{K}\).

El \textbf{conjunto de Fatou}, asociado a \(f,\) lo cual se denota como \(\mathcal{F}=\mathcal{F}(F)\) es aquel conjunto de todos
los \(z\in\mathbb{C}_\infty\), para los cuales la secuencia definida por inducción
por \(z_{0}=z,z_{n+1}=f\left(z_{n}\right)\), \textbf{no converge a infinito} \protect\hyperlink{plano1}{}

\hypertarget{plano1}{%
\section{\texorpdfstring{Plano de parámetros generado por \(Q_{p,c}=z ^p+c\)}{Plano de parámetros generado por Q\_\{p,c\}=z \^{}p+c}}\label{plano1}}

Es un conjunto de puntos llamados \textbf{parámetros},
definida por una función compleja \(f\), que contiene un parámetro
\(c\), es decir \(f_{c}(z)\), donde viven los conjuntos acotados (Multibrots)
\[
\mathcal{Q}:=\left\{c\in\mathbb{C}:\left\{f_{c}^{m}(0)\right\}_{m=1}^{\infty}\text{ es acotada}\right\}
\]
\[
\mathcal{M}^p:=\left\{c\in\mathbb{C}:\left\{Q_{c,p}^{m}(0)\right\}_{m=1}^{\infty}\text{ es acotada}\right\}
\]
de manera que este conjunto, divide al plano de parámetros (\(\mathcal{C}_\infty\))
en dos conjuntos complementarios \(\mathcal{Q}\) y \(\mathbb{C}_{\infty}\setminus\mathcal{Q}\).
\protect\hyperlink{plano2}{}

\hypertarget{plano2}{%
\section{\texorpdfstring{Plano dinámico generado por \(Q_{p,c}=z ^p+c\) (\(\mathcal{C}_\infty\))}{Plano dinámico generado por Q\_\{p,c\}=z \^{}p+c (\textbackslash{}mathcal\{C\}\_\textbackslash{}infty)}}\label{plano2}}

Este plano esta estructurado con los \textbf{conjuntos de Julia} cuyas propiedades dependen de la \textbf{ubicación del parámetro c} en relación a los conjuntos Multibrot, mencionado en la definición del plano de parámetros. \protect\hyperlink{www}{C3}

\hypertarget{www}{%
\chapter{Relaciones topólogicas entre los conjuntos Multibrot y Julia}\label{www}}

\bibliography{book.bib,packages.bib}

\end{document}
